\documentclass[11pt]{article}

% Change "review" to "final" to generate the final (sometimes called camera-ready) version.
% Change to "preprint" to generate a non-anonymous version with page numbers.
\usepackage[final]{acl}

% Standard package includes
\usepackage{times}
\usepackage{latexsym}

% For proper rendering and hyphenation of words containing Latin characters (including in bib files)
\usepackage[T1]{fontenc}
% For Vietnamese characters
% \usepackage[T5]{fontenc}
% See https://www.latex-project.org/help/documentation/encguide.pdf for other character sets

% This assumes your files are encoded as UTF8
\usepackage[utf8]{inputenc}

% This is not strictly necessary, and may be commented out,
% but it will improve the layout of the manuscript,
% and will typically save some space.
\usepackage{microtype}

% This is also not strictly necessary, and may be commented out.
% However, it will improve the aesthetics of text in
% the typewriter font.
\usepackage{inconsolata}

% Provide citation commands normally available via the ACL style.
% Loading natbib gives \citep, \citet, \citealp, etc. The \citeposs
% command is not standard, so we provide a simple fallback.
\usepackage{natbib}
\usepackage{url}
\providecommand{\citeposs}[1]{\citeauthor{#1}'s~(\citeyear{#1})}

%Including images in your LaTeX document requires adding
%additional package(s)
\usepackage{graphicx}

% Provide example images like `example-image-a` used below
\usepackage{mwe}

% If the title and author information does not fit in the area allocated, uncomment the following
%
%\setlength\titlebox{<dim>}
%
% and set <dim> to something 5cm or larger.

\title{Generate Image Base On Flow Matching }

% Author information can be set in various styles:
% For several authors from the same institution:
% \author{Author 1 \and ... \and Author n \\
%         Address line \\ ... \\ Address line}
% if the names do not fit well on one line use
%         Author 1 \\ {\bf Author 2} \\ ... \\ {\bf Author n} \\
% For authors from different institutions:
% \author{Author 1 \\ Address line \\  ... \\ Address line
%         \And  ... \And
%         Author n \\ Address line \\ ... \\ Address line}
% To start a separate ``row'' of authors use \AND, as in
% \author{Author 1 \\ Address line \\  ... \\ Address line
%         \AND
%         Author 2 \\ Address line \\ ... \\ Address line \And
%         Author 3 \\ Address line \\ ... \\ Address line}
\author{ZHONGHONG CHEN \\
  \texttt{MC55076@um.edu.mo} \\\And
  Second Author \\
  Affiliation / Address line 1 \\
  Affiliation / Address line 2 \\
  Affiliation / Address line 3 \\
  \texttt{email@domain} \\}

%\author{
%  \textbf{First Author\textsuperscript{1}},
%  \textbf{Second Author\textsuperscript{1,2}},
%  \textbf{Third T. Author\textsuperscript{1}},
%  \textbf{Fourth Author\textsuperscript{1}},
%\\
%  \textbf{Fifth Author\textsuperscript{1,2}},
%  \textbf{Sixth Author\textsuperscript{1}},
%  \textbf{Seventh Author\textsuperscript{1}},
%  \textbf{Eighth Author \textsuperscript{1,2,3,4}},
%\\
%  \textbf{Ninth Author\textsuperscript{1}},
%  \textbf{Tenth Author\textsuperscript{1}},
%  \textbf{Eleventh E. Author\textsuperscript{1,2,3,4,5}},
%  \textbf{Twelfth Author\textsuperscript{1}},
%\\
%  \textbf{Thirteenth Author\textsuperscript{3}},
%  \textbf{Fourteenth F. Author\textsuperscript{2,4}},
%  \textbf{Fifteenth Author\textsuperscript{1}},
%  \textbf{Sixteenth Author\textsuperscript{1}},
%\\
%  \textbf{Seventeenth S. Author\textsuperscript{4,5}},
%  \textbf{Eighteenth Author\textsuperscript{3,4}},
%  \textbf{Nineteenth N. Author\textsuperscript{2,5}},
%  \textbf{Twentieth Author\textsuperscript{1}}
%\\
%\\
%  \textsuperscript{1}Affiliation 1,
%  \textsuperscript{2}Affiliation 2,
%  \textsuperscript{3}Affiliation 3,
%  \textsuperscript{4}Affiliation 4,
%  \textsuperscript{5}Affiliation 5
%\\
%  \small{
%    \textbf{Correspondence:} \href{mailto:email@domain}{email@domain}
%  }
%}

\begin{document}
\maketitle
\begin{abstract}
This document is a supplement to the general instructions for
ACL proceedings, and contains instructions for using the
\LaTeX{} style file.
\end{abstract}

\section{Introduction}

% understand the math behind Image generation 

\subsection{Image Generation is Sampling}

For computer , a picture is usually represented as a matrix of pixels. 
For instance, a 16x16 handwritten digit image can be represented as a 16x16 matrix, where each element's value denotes the corresponding pixel's grey scale value (0–255).
This two-dimensional matrix can be flattened into a 256-dimensional vector $x \in \mathbb{R}^{256}$. 
In this case, one class of images ,such as the handwritten digit 3 , can be regarded as a data distribution within a high-dimensional space.
And any specific image instance is actually a point sampled from this distribution. 

\begin{figure}[h]
    \centering
    \includegraphics[width=0.4\textwidth]{images/figure1.png}
    \caption{MNIST dataset samples of digit 3, 4, 8}
    \label{fig:mnist_samples}
\end{figure}


Taking the MNIST dataset (downsampled to 16×16) as an example, all digits 3, 4, and 8 within the dataset belong to distinct distributions. By statistically analysing their grayscale mean values across different positions, the distribution characteristics of each digit can be derived. See Figure~\ref{fig:mnist_samples} for details.
This figure clearly demonstrates that the pixel distributions of digits 3, 4, and 8 differ across positions.
The data has been normalised here, with higher values indicating greater pixel intensity at that location, represented by darker shades.
What's more, a picture created by applying the mean grayscale values at each positions 
looks like the digit itself, as shown in Figure~\ref{fig:mnist_mean}.

\begin{figure}[h]
    \centering
    \includegraphics[width=0.4\textwidth]{images/figure2.png}
    \caption{Mean images of digit 3, 4, 8 in MNIST dataset}
    \label{fig:mnist_mean}
\end{figure}
 
In this situation, image generation can be viewed as a sampling a vector from a specific distribution in high-dimensional space.
And the vector sampled more close to the distribution's mean will generate an image more similar to the digit itself.
In the perspective of statistic , a vector sampled have a higher probobility belong to the "handwritten digit 3" distribution ,
it may looks more likely similar to a handwritten digit 3.


\section {Flow Matching Framework}

The idea of using neural networks to generate images is using neural networks to learn the transformation from a simple distribution (e.g., Gaussian distribution) 
to the target image distribution.

Flow Matching is a idea that modeling the transform between two distributions as a physical process , like particles flowing in a fluid. 
Under this idea,  a neural network is trained to learn the speed and direction of "particles" moving during the transformation space.

Based on physical knowledge, particle's position can be obtained by integration the velocity field. This integration process is equivalent to the sampling process.
This approach shares some similarities with diffusion models, which also achieve sampling from noise to images by learning an inverse process.
However, diffusion models accomplish sampling by learning a stochastic process, whereas flow matching models learn a smooth, physically meaningful velocity field.
The latter approach follows a more explicit path,achieving equivalent results typically requires fewer sampling steps compared to diffusion models.


\section{Experiments}

To better understand the training process under flow matching framework, 
We conduct a experiments on a custom dataset and
 observe how the model learns the transformation from Gaussian noise to target image distribution.

\subsection{Dataset}


\section*{Author Contributions}

ZHONGHONG CHEN: experiments. \\

\end{document}